%
% File acl2014.tex
%

\documentclass[11pt]{article}
\usepackage{framed}
\usepackage{acl2014}
\usepackage{times}
\usepackage{url}
\usepackage{latexsym}
\usepackage{graphicx,xcolor}
\usepackage{listings}
\usepackage{tabularx}
\usepackage{svg}
\usepackage{amsmath}
\usepackage{tikz}
\usepackage{tikz-dependency}
\usetikzlibrary{%
  shapes,%
  arrows,%
  positioning,%
  calc,%
  automata%
}
\definecolor{pf7}{RGB}{166, 118, 29}
\definecolor{grey}{RGB}{250, 100, 100}


\newcommand{\specialcell}[2][c]{%
  \begin{tabular}[#1]{@{}c@{}}#2\end{tabular}}

\def\subtitle#1\par{\removelastskip\bigskip\hrule
   \noindent\vrule height12pt depth5pt width0pt #1\par
   \hrule\nobreak\medskip
}
\def\begalg{\removelastskip\medskip
  \bgroup \parindent=0pt \tt \obeylines \obeyspaces  
          \everymath{\catcode`\ =10 \catcode`\^^M=5 }
}
\def\endalg{\egroup\medskip} 
{\obeyspaces\global\def {\ }}

% \renewcommand{\arraystretch}{1.5}
\title{Exploring the limits of shallow approaches on the MCTest}
% Building upon a lexical system for solving the MCTest ?? or something... ~ellery
\author
       {T. Brown, N. Greco, G. Mocanu and E. Smith
       \\
       Department of Computer Science\\
	University College London\\
       \tt{\{t.brown,n.greco,g.mocanu,e.smith\}@cs.ucl.ac.uk}\\ 
       }
\date{}

\begin{document}
\maketitle

\begin{abstract}
MCTest is a recently developed task for evaluating machine comprehension of text. We aim at approaching the task with shallow methods and a rule-based systems.
We start with a simple token matching baseline and we enhance it through additional pre-processing (i.e. co-reference resolution, hypernymy).
Using combinations of lexical features, we train a scoring function to score multiple-choice answers.
Introducing a simple matching rules system, we outperform the current systems by 4\% and 1\% in both the given datasets. Then we use the rule-based system to analyse the MCTest datasets, by evaluating the impact on performance of each rule.
\end{abstract}

\begin{figure}[!th]
\begin{framed}
\begin{flushleft}
{\small Q0}: It was Jessie Bear's birthday. She was having a party.  She asked her two best friends to
come to the party.  She made a big cake, and hung up some balloons.*\newline\newline
1) Who was having a birthday?\\
\textbf{A) Jessie Bear}\\
B) no one\\
C) Lion\\
D) Tiger\newline\newline
3) How did Jessie get ready for the party?\\
A) made cake and juice.\\
\textbf{B) made cake and hung balloons.}\\
C) made juice and cookies.\\
D) made juice and hung balloons.\newline\newline
{\small  \textsuperscript{*}Only relevant part of question {\small Q0} has been reported} 
\end{flushleft}
\end{framed}
\caption{\label{fig:exampleMCT} Story Q0 extracted from the MC160 development set}
\end{figure}

\section{Introduction}
% Within the field of NLP, techniques are often evaluated using narrowly defined tasks (such as recognising textual entailment, information extraction and semantic role labelling), that attempt to quantify progress in a specific subdomain of the task of the machine comprehension of text (MCT). This makes direct comparisons between a wide range of NLP systems somewhat prohibitive, and as such could well hinder the transfer of knowledge from the academic use of these techniques with controlled datasets, to the real world application of the research.

% MCTest \cite{mctest} is a task that allows the evaluation of techniques from multiple subdomains using the high-level goal of MCT, enabling more diverse comparisons across the field of NLP to be made and providing an assessment of the field's progress towards achieving real world MCT. 

Machine comprehension of text is a central goal in NLP. The academic community has proposed different challenges to measure progress, stimulating research in several fields: information extraction, semantic parsing and textual entailment. Yet these challenges often evaluate how we perform on individual techniques rather than how far we are from machine comprehension of text. A recent contribution proposes MCTest \cite{mctest}, a new challenge that aims at evaluating machine comprehension through an open-domain question answering task, requiring the common sense reasoning typical of a 7 year-old child. 

Our initial approach to tackling MCTest incorporates a simple bag of words algorithm as a baseline, which is then enhanced by adding pre-processing features such as co-reference resolution and finding hypernymy relations in the text. We then ensemble multiple approaches using various pre-processing features and in addition, use different variations of the original baseline algorithm, along with the use of a sentence selection algorithm to further improve the accuracy of the baseline. A linear SVM is then trained using the aforementioned features to classify a question-answer pair according to whether it is correct.

In order to analyse the MCTest dataset, we then construct a second, rule-based system that categorises the question-answer pairs and subsequently applies transformations to the story text. Words within the text are then weighted depending on the question category, and then passed to a scoring function based upon the Sliding Window with Distance algorithm seen in \cite{mctest}. This system allows us to gain a quantitative understanding of the dataset, by allowing us to see the individual effects of each component of the system on a question by question basis. Using the data that this generates, we evaluate the relative performances that various features have on the dataset, then also analyse the differences between the MC160 and MC500 collections as well as the type of questions that appear in each collection.

\section{Previous work}
The use of shallow methods for the machine comprehension of text has been well explored in previous work. Using a bag of words algorithm for textual entailment tasks was pioneered by \cite{deepread}, this matches question-answers pairs to sentences in the text and chooses the best pair with the best matching sentence. This algorithm counts the words in common between the pair and the sentence and uses that as the metric to decide on the answer. These systems ignore predicate-argument structure and can easily fail in the presence of quantifiers, negations or synonyms. Furthermore in \newcite{behaviour_best} such methods are classified as being ``cheap tricks" that are deceptive, as they do not prove the true comprehension of text.

Other work combines the bag of words algorithm with more complex means of relating questions to the answers by representing each pair as vectors and them evaluating them in a shared vector space \cite{deep_selection}. In this case the bag of words is used to construct the vector representation of a sentence. The result is a vector with the number of mentions for each individual word in that sentence, which is then normalized by the length of the sentence. In the same paper they show that using a bigram model created with a convolutional neural network is superior to the bag of words model.

The bag of words model can also be used to generate a feature space for a scoring function \cite{memory_networks}. This can be made to create multidimensional feature spaces by having each dimension represent a different sentence that we want to examine. Such a function can be used to train a neural network to match facts with sentences and produce an answer for a question. In the case of an unseen word then the system would also use bag of words to store the context in which that word was used, generating a left and right bag.  

Apart from the above bag of words approaches, other work focuses on the use of feature engineering. For example, \newcite{learning_entailment} designs a set of 28 features and then train a classifier to determine textual entailment. Examples of the features included are: polarity, antonymy, modality, quantifiers, numerical and temporal, etc. Textual alignment can be used to automatically extract features as demonstrated in \cite{relation_alignment}, which has the advantage of being adaptive as opposed to having static features. 

More complex methods which strive to derive deeper knowledge than the shallow methods used in our system make use of Natural Logic as in \cite{mccartney2007manning}. This decomposes the problem into a set of small local edits, which transform the hypothesis into the premise. In practice, the hypothesis is the question answer pair and the premise is a sentence from the text the question refers to. \newcite{angeli2014naturalli} combines Natural Logic with a distributional model to obtain a likely valid derivation (with a confidence interval) and to compute for multiple premises simultaneously. 

Our work explores the limits of shallow methods and also how they compare with each other. As opposed to most work focusing only on one method and refining it, we use a broad spectrum of different improvements to a basic bag of words baseline as well as build our own heuristic system. 

\section{Task description}
The MCTest dataset consists of two collections of fictional stories, MC160 and MC500, containing 160 and 500 stories respectively with 4 questions per story and 4 multiple choice answers for each question, all of which were written by workers using Amazon Mechanical Turk.  The first collection of stories (MC160) was manually curated for quality by \cite{mctest}, whereas the second collection (MC500) was constructed by workers who had been pre-screened using a grammatical test and the stories were then rated for quality by new workers. MC500 also stipulated that workers include two incorrect answers in the story, to ensure that questions are not trivially answerable. The dataset is open-domain, however the concepts and vocabulary used within the stories and questions are limited to those which are understandable by a 7 year old, hence they may have fictional settings (e.g. {\em giraffes going to parties}) and require common sense knowledge.

\section{Scoring function}
We can model the challenge in the following way: assuming passage $P$ has a set of questions $Q$,  where each question $Q_i \in Q$ is associated with a list of answers $\{A_{i1}, A_{i2}, ..., A_{im}\}$ (in MCTest $m=4$) with associated judgement, $\{y_{i1}, y_{i2}, ..., y_{im}\}$ where $y_{ij}=1$ if the answer is correct, otherwise $y_{ij}=0$. Our task is to assign a score to a triple 

\begin{equation}
S_{ij}=Score(P, q_{i}, a_{ij})
\end{equation}

and choose the triple with the maximum score (or more, in case of equal score) as the chosen answer(s), $A_{i}$.

\begin{equation}
A_{i} = \{i : \max_{j=0}^{m}(S_{ij})\}
\end{equation}

Our approach is to construct a simple baseline scoring function and improve upon it. We will build a set of scoring function and combine their results to finally learn a new scoring function through a classifier over these triples so we can predict the judgment of other QA pairs.

    \subsection{Token-matching baseline}
    We propose a baseline scoring function, a simpler version compared to system proposed by \newcite{mctest} that only uses token matching (algorithm in Figure~\ref{fig:algoBOW}). Our system matches a bag-of-words constructed from the question and a candidate answer with each sentence in the story. The question-answer pair with the most words in common with a sentence in the text is considered the best candidate answer. Normalization are applied such as removing stop words stemming to remove affixes from words.

\begin{figure}[!th]
\subtitle Definitions

Passage $P$, word tokens in $j$-th sentence $P_j$, word tokens in question $Q$, word tokens in candidate answers $A_{1\ldots4}$ and set of stop words $X$.

\subtitle Algorithm 1 Sentence level bag-of-words

\begalg
def $TokenMatchingScore(P, Q, A)$:
  $score$ = $0$
  for $j$ = $0\ldots|P|-1$:
    $tmp$ = $(A_i\cup Q) \cap P_j \setminus X$
    if $tmp$ > $score$:
      $score$ = $tmp$
  return $score$
\endalg
\caption{\label{fig:algoBOW} Lexical-based baseline algorithm }
\end{figure}

This semantic overlap approach treats the problem as a textual similarity task and would perform best when the pair is a subset of a sentence. However, this strategy ignores predicate-argument structure and can easily fail in the presence of quantifiers, negations or synonyms. Work on story comprehension using bag-of-words has a long history, \newcite{deepread} proposed D{\small EEP}R{\small EAD} and showed how such systems with some heuristics can achieve high accuracy especially on questions with ``who / what / when", which is most part of our questions.

Results for MC160 and MC500 are shown in Table \ref{table:resultBOW}. The token-matching baseline has been authored without seeing both the test sets. The results are split for questions that need one sentence (``Single") or more sentence (``Multi") to answer the question. The 15\% drop between ``Single" and ``Multi" is due to choosing the most matching sentence in the text, hence multiple-sentence questions are clearly disadvantaged.

\begin{table}[!th]
\centering
\begin{tabular}{c|c|c|}
\cline{2-3}
\textbf{}                             & \textbf{MC160} & \textbf{MC500} \\ \hline
\multicolumn{1}{|c|}{\textit{Single}} & 67.88\%        & 62.763\%        \\ \hline
\multicolumn{1}{|c|}{\textit{Multi}}  & 50.31\%        & 46.72\%        \\ \hline
\multicolumn{1}{|c|}{\textit{All}}    & 58.43\%        & 53.97\%        \\ \hline
\end{tabular}
\caption{\label{table:resultBOW}Dev+train results for the bag-of-words baseline augmented with hypernym}
\end{table}


    \subsection{Sentence selection}
    % Selecting the sentence with the most words in common is
% is a clear disadvantage, since MCTest has questions whose answer may be in contained in multiple sentences, and, running matching the bag of words between the entire passage and the question-answer pair would score poorly, since this would score how similar is the QA pair to the story on the basis of term frequency.

To improve on answering questions that need multiple sentences, we propose to run the bag-of-words between each question-answer pair and the $n$ most relevant sentences for the question; such problem reduces to ``Learning to Rank" task. Previous attempts of retrieving the sentences with the highest relevance for question answering have been already proved successful in the literature \cite{qa_techniques,deep_selection}. %TODO better context?
The process consist in retrieving the top $n$ ranked documents, (in this case, sentences) for a query $Q$. All the sentences $S_i$ in the story $S$ are scored by a scoring function $SentenceScore(Q, S_i)$ and then ranked.

Choosing the right query for sentence retrieval is as important as the implementation of the scoring function. Initially, our query was a stemmed and stop-words cleaned version of the question. Following an evaluation on the MC160 development set, we found that in some cases the sentences retrieved were relevant to the answer, but not to the question. For example (see Figure~\ref{fig:exampleMCT}), {\em ``Who was having a birthday?"} contains the right keyword {\em ``birthday"} to understand that the first sentence in the passage contains the answer, however {\em ``How did Jessie get ready for the party?"} does not. Sentences like {\em ``She was having a party"} will very likely have an higher score than {\em ``She made a big cake, and hung up some balloons."} which exactly contains the correct answer ($B$). Hence, we decided to also consider a combination of question and answer as alternative to question-only query.

It is beyond the interest of this paper to improve on ranking relevant sentences; hence, we use our $TokenMatchingScore$ to score each sentence and combination of it with pre-processing (in detail in Section~\ref{sec:combined})

    %\subsection{Pre-processing}
    % Matching question-answer pairs with the story can be significantly improved by homogenising the format of all stories and question-answer strings. Our matching algorithms operate on raw textual tokens, which are lemmatized and stripped of all extraneous function words; however, this raw format is generated on-demand, rather than during the pre-processing stage, and we retained the deep grammatical structure of the text in order to dynamically alter the format based on certain question conditions.
%transform the two to become more {\em similar} to each other and removing words that are noise in the text.
%Previously discussed pre-processing of the text was removing stop words, tokenizing the text into sentences and into words, stemming words, and adding POS notation for which we used the Stanford Parser\footnote{Some note on the Stanford parser}
%TODO

We focused on three pre-processing stages: syntactic parsing and co-reference resolution, hypernym and synonym annotation and answer statement generation.

    \begin{figure*}[!th]
    \begin{dependency}
        \begin{deptext}[column sep=0.1em, row  sep=0.2em]
            (1) \&
            Peter \&
            was \&
            a \&
            very \&
            sad \&
            puppy.\&
            He \&
            had \&
            been \&
            inside \&
            of \&
            the \&
            pet \&
            store \\ \hline
            (2) \&
            peter \&
            \textcolor{grey}{be} \&
            \textcolor{grey}{a} \&
            \textcolor{grey}{very} \&
            sad \&
            puppy. \&
            peter \&
            \textcolor{grey}{have} \&
            \textcolor{grey}{be} \&
            inside \&
            \textcolor{grey}{of} \&
            \textcolor{grey}{the} \&
            pet \&
            store  \\ \hline
            (3) \&
             \&
             \&
             \&
             \&
             \&
            (dog, 1.0) \&
             \&
             \&
             \&
             \&
             \&
             \&
            \&
            (shop, 1.0)  \\
            \&
             \&
             \&
             \&
             \&
             \&
            (canine, 0.5) \&
             \&
             \&
             \&
             \&
             \&
             \&
            \&
            (establishment, 1.0)  \\
            \&
             \&
             \&
             \&
             \&
             \&
            (animal, 0.3) \&
             \&
             \&
             \&
             \&
             \&
             \&
            \&
            (place, 0.5)  \\
            \&
             \&
             \&
             \&
             \&
             \&
            (young, 0.5) \&
             \&
             \&
             \&
             \&
             \&
             \&
            \&
            (business, 0.5)  \\
            \&
             \&
             \&
             \&
             \&
             \&
            .. \&
             \&
             \&
             \&
             \&
             \&
             \&
            \&
            ..  \\
        \end{deptext}
        \depedge[edge unit distance=0.2ex]{8}{2}{\bf\textcolor{pf7}{co-ref}}
    \end{dependency}

    \caption{\label{fig:preprocessing}Examples of different transformations: (1) The initial sentence with co-reference annotation, (2) Each token is lemmatised and stop words are marked in red, (3) Hypernym for each token associated with their score }
    \end{figure*}

        % TODO
        % \subsubsection{Syntactic Analysis}
        % The initial pre-processing stage used the Stanford Parser \cite{stanford_parser} and Stanford Dependencies \cite{de2008stanford} to obtain phrase-structure and dependency trees for each story, along with its questions and answers. We also made use of the lemmatization and Penn part-of-speech (POS) tagging \cite{marcus1993building} provided by this system.

This toolkit performed well on the given data, due to the intentional linguistic simplicity of the stories. Of the few inconsistencies, the majority were due to incorrect recognition of invented brand names (e.g. ``Cookies n' Cr\'{e}me'' and ``Friendly-O's'') and the inability to categorize some subordinate dependencies. However, such cases were rare, and the errors introduced at this pre-processing stage were negligible in the final results.

        \subsection{Co-reference resolution}
        Co-reference refers to extracting relations between expressions that refer to the same entity. In most cases when we have a co-reference relation one of the expressions is the full form of the referred object, called the antecedent and the other is the abbreviated form, which can take many forms. For our system having this information is useful as we can replace the abbreviated form with the full form and obtain more useful sentences to match against. 

The co-reference information was extracted using the Stanford CoreNLP \cite{manning2014stanford}. The passage text was parsed independently of the question and answer strings, so all co-reference chains were local to the story itself. We used an out-of-the-box configuration of the co-reference rules, as this was deemed to perform adequately on the simplistic format of the given stories. Using co-reference on the question answer pairs, or on the story text plus the question answer pairs proved to be detrimental to performance.

        \begin{table}[!th]
        \centering
        \begin{tabular}{c|c|c|}
\cline{2-3}
\textbf{}                             & \textbf{MC160} & \textbf{MC500} \\ \hline
\multicolumn{1}{|c|}{\textit{Single}} & 70.54\%        & 62.82\%        \\ \hline
\multicolumn{1}{|c|}{\textit{Multi}}  & 54.07\%        & 49.39\%        \\ \hline
\multicolumn{1}{|c|}{\textit{All}}    & 61.69\%        & 55.46\%        \\ \hline
\end{tabular}

        \caption{\label{table:resultBOWcoref}BOW with coref}
        \end{table}

        \subsection{Hypernymy and Synonymy}
        % Ellery X

% version1
%After the story text had been tokenized and annotated, we used the WordNet \cite{miller1995wordnet} interface provided by the Natural Language Toolkit \cite{bird2006nltk} to obtain basic semantic information for each token in the passage. During this stage, each token was paired with a subset of its WordNet synset tree.

%After analysing patterns in the format of the question strings, we opted to consider only synonymy and hypernymy information as part of this task. Questions would typically contain a general word describing a certain category of object (e.g. "What animal...?"), and the text itself would contain a more specific reference to an entity (e.g. "Peter the puppy..."). By expanding the words in the story text with hypernymy information, a relation can be extracted between the entity reference in the question, and the mention in the passage (as shown in the example (3) in Figure~\ref{fig:preprocessing}). 

%version2
During the error analysis of our baseline system, we found that a number of questions would contain a general word describing a certain category of object (e.g. "What animal...?"), and the text itself would contain a more specific reference to an entity (e.g. "Jeff the dog..."). In such cases, lexical overlap between the question and the story is low, and our system struggled to locate the relevant portion of the text. However, if we view the words in a question as super-ordinates or co-ordinates of the words in the story, we can match words with a \textit{type-of} relation, in addition to lexical overlap.

Based on the semantic substitution system of \newcite{angeli2014naturalli}, we aimed to augment our existing bag-of-words system by providing more generalized alternatives for certain words in the story text. To do this, we used the WordNet \cite{miller1995wordnet} interface provided by the Natural Language Toolkit \cite{bird2006nltk} to annotate each token in the story with synonymy and hypernymy information. However, rather than attempting to extract semantic relations using a token's synset tree, which is computationally expensive, we simply expand each story word to a series of hypernyms during the preprocessing stage. This has the advantage of being easy to integrate into our existing bag-of-words system, and the results of this augmentation are shown in Table \ref{fig:resultHypDev}. We observed a significant improvement on the MC500 stories, however, this addition was in fact detrimental to accuracy on the MC160.

\begin{table}[h]
\centering
\begin{tabular}{c|c|c|}
\cline{2-3}
\textbf{}                             & \textbf{MC160} & \textbf{MC500} \\ \hline
\multicolumn{1}{|c|}{\textit{Single}} & 71.53\%        & 62.93\%        \\ \hline
\multicolumn{1}{|c|}{\textit{Multi}}  & 50.04\%        & 49.28\%        \\ \hline
\multicolumn{1}{|c|}{\textit{All}}    & 59.98\%        & 55.45\%        \\ \hline
\end{tabular}
\caption{\label{fig:resultHypDev} Dev+train results for the bag-of-words baseline augmented with hypernymy}
\end{table}

Synonymy, however, was used only as an experimental tool, and was not included in the final system. We hypothesised that, since these stories were written with an audience of young children in mind, and as such contained a relatively small lexicon, synonymy would merely produce additional noise. When it was included, any changes in our results varied significantly from one dataset to another, and selecting the most common sense of a word provided more accurate and stable results, thus confirming our hypothesis.

        % \subsubsection{Question answer combiner}
        % Using the tokenized and annotated questions and answers from the MCTest dataset, we also created a system to automatically generate statements from question-answer pairs that would be useful for approaching the task as a recognizing textual entailment problem, to allow for feature extraction using the statements later in the system. We take a similar approach to the search engine query reformulation seen in \cite{brill02ananalysis}, but additionally utilise the benefit of having parsed text with part of speech tagging.

Our system uses multiple string-based manipulations to rewrite the question-answer pair into various different forms that could then be used as a hypothesis for evaluation. Rewrite rules are applied to the question-answer pair depending on the classification of the type of question, as well as the lexical structure of the answer.

Each possible rewrite is then evaluated using a language model trained using the SRILM toolkit \cite{Stolcke2002} with training data obtained from English Wikipedia. The system uses a 5-gram language model to assign a probability of occurrence to each possible rewrite and then the system chooses the rewrite with the highest probability to then be used for feature extraction in later parts of the system. 


\subsection{Combining shallow methods}
\label{sec:combined}

We improved the simple token-matching baseline combining it with pre-processing techniques and sentence selection. Each of the baseline achieves different results. 

\begin{itemize}
\item \textbf{Baseline+COREF} \\
We
\item \textbf{Hyperbow}
\item \textbf{BOWALL+COREF with sentence selection using BOWALL+COREF on q}
\item \textbf{BOWALL+COREF with sentence selection using BOWALL+COREF on qa}
\item \textbf{HYPERBOW with sentence selection with BOWALL+COREF on qa}
\item \textbf{BOWALL with  with sentence selection using HYPERBOW on QA}
\end{itemize}

After an analysis of the results in the test set, we found that some baseline predict the same score to multiple answers, hence combining the different methods we can gain %TODO something

\begin{table}[!th]
\begin{tabular}{|l|c|c|}
\hline
 & MC160 & MC500 \\ \hline
TM+Co-reference   & 61.69\%  & 55.46\%  \\ \hline
TM+Hypernym   & 59.98\%  & 55.45\%  \\ \hline
TM+Coref (selection Q)   & 57.50\%  & 51.30\%  \\ \hline
TM+Coref (selection QA)   & 56.81\%  & 44.32\%  \\ \hline
TM+Hypernym (selection QA).   & 59.42\%  & 55.36\%  \\ \hline
TM+Coref (s. hypernym QA).   & 57.46\%  & 43.28\%  \\ \hline
\end{tabular}
\caption{\label{table:results}Percent correct on MC160 and MC500 dev+train sets.}
\end{table}


In order to build such classifier we built a set of features that combine bag of words with sentence selection, hypernymy, co-reference, we trained a linear SVM to estimate the class of each sentence. Given that we model we described, in order to associate a score to the binary classification, we calibrate the probability of being part of a class, using Platt scaling \cite{plath_scaling} - logistic regression applied on SVM's score, these are fit by a further cross-validation on the training data. Finally we do parameter optimization through grid search.

\section{Rule-based system}
\label{sec:rulebased}
% Ellery X

In addition to the previous methods, we also developed a rule-based scoring system. This applies a series of transformations to the story text, and assigns weights to each word, based on the conditions required to answer a given question. For example, questions requiring negation will cause the weight value for each word to be inverted, since we will need to search for the answer that is least likely to be inferred by the story.

Each rule consists of a syntactic or lexical pattern, which filters question-answer pairs into categories, according to certain empirically-designed criteria. If a question falls into one or more of these categories, a corresponding story transformation procedure is invoked. Each procedure may add or remove words from the story, or alter the weight value associated with each word. There are several base transformations applied for all questions, including lemmatization and removal of stopwords.

Some examples of our rule-based question filters are:

\textit{Negation}. Negation covers questions such as ``What did James not eat for dinner?''. These questions are a distinct special-case, which cannot be solved by simply selecting the answer with the highest similarity to the story, as all of our other methods do. We detected negative questions using the syntactic dependency graph of the question string: a question is negative if and only if the sentential root is linked to any other node by a negated dependency.

\textit{Character-subject}. Character-subject questions are those which directly concern the actions of a named character, such as ``Why did Jon go to the park?''. Any question with a named entity as a nominal-subject dependent of the question's head verb was deemed to be a character-subject question.

\textit{Narrative}. Narrative questions are those which pertain to the structure of the story itself, for example ``Which character was first mentioned in the story?''. These questions necessitate that a given system understands the concept of a story, which will have characters and a defined narrative flow, rather than simply viewing it as an unstructured piece of text. We detected these instances using a series of cue-words, such as ``story'', ``passage'' or ``character''. These cues indicate that the question acknowledges the concept of a story.

\textit{Temporal}.

\textit{Implicative}.



% In addition to the previous methods, we also developed a rule-based scoring system. This applies a series of transformations to the story text, and assigns weights to each word, based on the conditions required to answer a given question. Each rule consists of a syntactic or lexical pattern, which filters question-answer pairs according to certain empirically designed criteria. Examples of such filters include negated questions, ``why'' questions, or questions requiring temporal recognition of events.

% If a question falls into one or more of these categories, a corresponding story transformation procedure is invoked. Each procedure may add or remove words from the raw textual output of the story, or alter the weight value associated with each word. There are several base transformations applied to all questions, including lemmatization, removal of stopwords, and inverse term frequency weighting.

% After these transformations have been applied for a given question-answer pair, the weighted tokens are then passed to a scoring function, such as the weighted bag-of-words used in Algorithm 2. We found that the best performance, however, came from utilising a modified version of the Sliding Window with Distance algorithm used in \cite{mctest} as our base scoring function. In combination with the rule-based system and SVM, we managed to exceed the performance of the algorithm in its original form. Additionally, in the case of the MC160, we were able to outperform the author's implementation of the same algorithm combined with a Recognising Textual Entailment system.


% While we registered an improved result, it should be noted that the improvements of the rule-based transformations were less significant when compared to the effect of the simpler transformations, such as lemmatization. However, it was not our intent to over-engineer these rules when it would be built upon a relatively shallow lexical baseline. As such, all transformations were somewhat superficial heuristic techniques. The primary goal of this implementation was to categorize and analyze the performance of a given scoring system, and to gain an insight into the flaws of both the system and the test itself, which are presented in the Evaluation section. Additionally, it does demonstrate a proof-of-concept for a deeper rule-based system, with more sophisticated transformation algorithms on top of a stronger baseline.


%
%
%\begin{figure}
%
%\subtitle Definitions
%
%Passage $P$, $W(w)$ weight of word $w$, question $Q$, answers $A_{1..4}$ and a set of rule-transformation pairs $(R,T)$.
%
%\subtitle Algorithm 2: Question filtering
%
%\begalg
%for each $(R,T)$ do
%  if $R(Q,A)$ holds then
%    $P = T(P)$
%end for\\
%for $i = 1\ldots4$ do
%  $S = (A_i \cup Q) \cap P$
%%$r_i = sc(P,S)$
%  $\displaystyle r_i = \sum_{j=0}^{\left |  S\right |} W_{S_j}$
%end for\\
%return $r_{1\ldots4}$
%\endalg
%
%\caption{\label{fig:rulebasedalg}Rule-based filtering algorithm }
%\end{figure}


\section{Experiments and results}

Our baseline under performs the MCTest baseline.

We combined the six BOW-based methods and we classify it with our SVM scoring function.

The rule-based system (RBS) outperforms the on the MC160 Test set, hence the SVM combined with the RBS. The SVM improves of about 5\% the RBS on the MC500, beating the MCTest baseline, but still underperforming compared to the RTE results.

\subsection{Combining RBS with shallow methods}

\subsection{Combining RBS with RTE}
To make our results comparable with the second baseline in \newcite{mctest}, we augmented our rule-based system with the state of the art RTE system BIUTEE \cite{biutee} with default settings. Even in this case, our combination outperforms their result 

\begin{table}[!th]
\begin{tabular}{|l|c|c|}
\hline
 & MC160 & MC500 \\ \hline
Baseline         & 54.90\%  & 50.72\%  \\ \hline
% TM+Co-reference   & 61.81\%  & 53.24\%  \\ \hline
% TM+Hypernym   & 58.75\%  & 51.82\%  \\ \hline
% TM+Coref (selection Q)   & 60.17\%  & 48.22\%  \\ \hline
% TM+Coref (selection QA)   & 57.92\%  & 42.60\%  \\ \hline
% TM+Hypernym (selection QA).   & 59.13\%  & 51.65\%  \\ \hline
% TM+Coref (s. hypernym QA).   & 56.46\%  & 42.88\%  \\ \hline
SVM  & 67.57\%  & 56.31\%  \\ \hline
\hline
RBS   & 71.77\%  & 58.67\%  \\ \hline
RBS + SVM   & 71.35\%  & 60.17\%  \\ \hline
RBS + RTE   & \textbf{73.50}\%   & \textbf{64.20}\% \\ \hline
\hline
MCTest (SW+D)   & 68.02\%  & 59.93\%  \\ \hline
MCTest (SW+D)+ RTE   & 69.27\%  & 63.33\%  \\ \hline
\end{tabular}
\caption{Percent correct for the multiple choice questions on MC160 and MC500 test sets}
\label{table:results}
\end{table}
% Nicola ? % Emil ? % Tom?
%1 page


%TODO this is to explain sentence selection
combined with hypernym and co-reference to rank the most relevant sentences (in details in Section~\ref{sec:bigmix}). By tuning on both the MC160 and MC500 training sets, we set $n=3$ as it performs best.

\section{Evaluation of strategies}
% Ellery X

Following our experiments on the test set, we then conducted an evaluation of our scoring functions and also used the rule-based system to analyse both the MCTest dataset and the impact on performance of individual components of our rule-based system.

\subsection{bow, svm, etc. evaluation}

% maybe this should go in conclusion?
In addition, although  scoring function built on the SVM achieves good %find another word
results, its five-fold cross-validation is computationally very expensive.

\subsection{Rule-based System}

Using our rule-based system, we were able to perform a quantitative analysis on the performance improvement or degradation for each individual component. In addition, we could view the state of the story text after each transformation had been applied, giving us an insight into the effects of each filter on a per-question basis. From this, we gained a greater understanding of the differences between the MC160 and the MC500 tests, and the performance of our optimal system on each individual question category.

\begin{table}[!th]
\begin{tabular}{|c|c|c|}
\hline
\textbf{Transformation} & \textbf{MC160} & \textbf{MC500} \\ \hline
\textit{Tokenization (baseline)} & 61.13\%                 & 51.76\%                 \\ \hline
\textit{Lemmatization}           & \textbf{+1.19\%}                 &\textbf{ +2.47\%}                 \\ \hline
\textit{Co-reference Resolution}             & -1.19\%               &  \textbf{ +2.43\%}                 \\ \hline
\textit{Stopword Removal}        &\textbf{ +0.44\%}               &  \textbf{ +1.30\%}                 \\ \hline
\textit{Hypernym Expansion}        & +0.69\%                 & +1.02\%                 \\ \hline
\textit{Inverse Token Frequency} & \textbf{ +3.00\%}                & \textbf{ +4.82\%}                 \\ \hline \hline
Optimal Combination     & \textbf{+6.94\%}        & \textbf{+5.79\%}        \\ \hline
\end{tabular}
\caption{Performance of base transformations on combined Development and Training sets. Optimal Combinations are in bold.}
\label{table:ruleres}
\end{table}

In Table~\ref{table:ruleres}, we present the accuracy scores for the Development and Training datasets, using the Sliding Window with Distance algorithm as a scoring function, along with the relative improvement of each base transformation. The base transformation routines are performed on all stories, regardless of the content of the question-answer pair. We determined the optimal combination of base transformations through observation, and this result is also displayed in Table~\ref{table:ruleres}. It is noteworthy that the optimal configuration differs from the MC160 to the MC500, in particular the absence of coreference resolution in the final MC160 result.

As would be expected, lemmatization provides a notable increase in performance, particularly in the MC500. The loss of information from matching only word stems is negated by homogenizing the format of the question, answer and story, and this relatively cheap improvement was used in all algorithms we have presented. A further inexpensive optimisation was the removal of stopwords, using a list obtained from the Natural Language Toolkit \cite{bird2006nltk}, which increased performance across all datasets. However, while we removed such words from the raw tokenized output of the story, and thus these words were not scored, they were still counted as part of the sliding window size and word distance in the scoring function. Conversely, any hypernyms added to the text were not considered when counting the distance between words.

For the hypernym expansion procedure, each grammatical token was mapped to an extended set of raw lexical tokens, based on the parent token's WordNet hypernym tree. Each additional token is weighted based on its depth in the given hypernym tree, such that less relevant relations have a smaller impact on scoring. While this did register an improvement over the baseline system, the additional words had very little impact, or were occasionally detrimental to performance, when combined with higher-impact methods, such as co-reference and token frequency. Most hypernyms, especially those further down a token's tree, were merely noise. In addition, this procedure is computationally expensive, and the low number of hypernym matches overall did not justify its inclusion in the optimal configuration.

Most notably, co-reference resolution decreases accuracy on the MC160 datasets, while having the opposite effect on the MC500. This may be due to the increased simplicity of the questions in the MC160 stories: in many cases, the correct answer string is taken verbatim from the story text, and as such, there is enough information in the unaltered story to identify the relevant sentences. For these questions, which are much more prevalent in the MC160 than in the MC500, resolving co-references will give a higher score to the incorrect answers, which introduces inaccuracies into an otherwise clear match. In addition, the mistakes inherent to even a state-of-the-art co-reference resolution system may provide misleading information. An example from the MC160 Test set would be ``Todd is a small boy from the town of Rocksville'' becoming ``Todd is Todd'' or ``Todd is Todd a small boy from the town of Rocksville''. In either case, a simpler system would suffice. However, the MC500 questions are more subtly designed, and benefit from the additional information provided by co-reference resolution.

Multiplying the weight of each word by the inverse logarithm of its frequency in a given story produces the most significant improvement of all base transformations, and is computationally insignificant when these frequencies are calculated during the pre-processing stage. This is a common technique used in Information Retrieval systems, though the document size is typically much larger than in this task, and requires no syntactic or semantic information. Interestingly, this lexical trick has a greater effect on accuracy than much deeper methods such as coreference.

\subsection{Question Analysis}
% ellery X

Using the question-filtering rules, we were able to obtain an individual accuracy score for each question category on our optimal baseline, and a comparison with the overall score is presented in Table \ref{table:catres}. Note that while we improved upon many of these scores in our final rule-based system, here we present the baseline scores in order to analyze the performance of a ``pure'' bag-of-words system.

\begin{table}[!th]
\centering
\begin{tabular}{c|c|c|}
\cline{2-3}
                                                 & \multicolumn{2}{|c|}{\textbf{Relative Accuracy}} \\ \hline
\multicolumn{1}{|c|}{\textbf{Category}} & \textbf{MC160}                 & \textbf{MC500}                \\ \hline
\multicolumn{1}{|c|}{\textit{Negation}}          & \textbf{-55.03\%}          & \textbf{-28.98\%}         \\ \hline
\multicolumn{1}{|c|}{\textit{"Why" Questions}}   & +1.26\%                    & +6.23\%                   \\ \hline
\multicolumn{1}{|c|}{\textit{Character Subject}}   & \textbf{-5.71\%}           & -1.65\%                   \\ \hline
\multicolumn{1}{|c|}{\textit{Narrative}}         & \textbf{-7.83\%}           & \textbf{-20.59\%}         \\ \hline
\multicolumn{1}{|c|}{\textit{Temporal}}          & -1.39\%                    & +5.69\%                   \\ \hline
\multicolumn{1}{|c|}{\textit{Implicative}}       & -3.66\%                    & +4.59\%                   \\ \hline
\end{tabular}
\caption{Performance of the optimal baseline on a selection of question categories, in comparison to the overall score. Notable difficult cases are highlighted.}
\label{table:catres}
\end{table}

There are several clear deficiencies in certain areas, particularly in handling negation. These errors provide a broad overview of the cases in which simple lexical techniques are not sufficient to determine the correct answer.

We detected negative questions using the syntactic dependency graph of the question string. A question is negative if and only if the sentential root is linked to any node by a negated dependency. We identified two primary types of negative questions: direct negation and indirect negation. Questions with direct negation are of the form ``What wasn't in the story?'', for example, and it is sufficient to locate the singular answer which is not referenced in the passage. Indirect negation encompasses all other questions with a negated headword.

Instances of direct negation are impossible to answer using a standard word-matching algorithm, without special provision for such cases. The MC160 contains a larger proportion of these questions, and thus our baseline registers better performance on the MC500, with fewer such cases.

However, these instances of simple negation are trivial to solve, and our final system implements a rule for such questions, and answers all correctly. However, the remaining negative questions, which are a significant majority in the MC500, have proven to be more difficult to engineer. We attempted a shallow heuristic transformation using the explicit negation cues from \newcite{councill2010s}, which registered minor improvements.

Additionally, we observed that, for questions which directly relate to the actions of a named entity, it is likely that the relevant instance will be referred to in pronominal form in the story text. To study these cases, we utilised syntactic dependencies, combined with part-of-speech tagging, to detect the presence of a proper noun subject for each question. It can be seen that, as consequence of removing co-reference from the MC160 baseline, questions directly concerning a named character under-perform, in comparison to the MC500. However, this is a necessary consequence in order to increase the overall performance on the MC160.

The performance of our system on more abstract questions, concerning the overall narrative of the story, also demonstrates a significant inadequacy of lexical-based algorithms. Questions such as ``What was the first character mentioned in the story?'', which relate to the overall narrative flow of the passage, and general questions concerning the state of the story environment, such as ``Where is the story set?'', are difficult to solve without a system which understands the concept of a story itself. Traditional question-answering methods would also struggle here, and it seems as though a model designed specifically for reading comprehension contexts is required to truly perform well at this task.

\subsection{MC160 and MC500 Comparison}
%ellery X

It is evident both from our results, and the original baseline, that the MC500 corpus is significantly more challenging than the MC160. However, this may be primarily due to the inferior design of the MC160, which was created prior to the MC500. Our observations show that the MC160 can be more easily beaten by so-called ``cheap tricks'', while the MC500 has proved more resilient to the same optimisations. The base perfomance of our rule-based system on the MC160, with no additional transformations, is almost 10\% more accurate than the same system running on the MC500. However, the MC160 registers smaller improvements, or even decreases, in accuracy when more sophisticated components, such as coreference and hypernymy, are added.

This is a consequence of the design and curation process of the MC500 corpus, which required that answers must not be contained directly within the story text, or, if they are, two or more misleading answers must also be included. This decision portrays the MC500 as a textual entailment problem, and when \newcite{mctest} utilised an RTE system in tandem with their lexical baseline, they noted a drastic increase in accuracy on the MC500, and a minor increase on the MC160.

However, in many cases, the MC500 appears to be over-engineered to confuse lexical algorithms, and to encourage the use of entailment processes, by including complex sequences of negation and inference, such as in Figure \ref{fig:exampleMC500hard}. But, from our observations, the MC500 is a more accurate measure of machine comprehension, and will require a system with genuine textual understanding to solve.

\begin{figure}[!th]
\begin{framed}
\begin{flushleft}
{\em Q:} Mrs. Frank took roll call. She seemed nice, and I'm happy we weren't told to sit in alphabetical order or by boys to boys and girls to girls, as I was free to sit by myself for now. Mrs. Frank called out Jimmy, Sally, Linda, Jason, and then finally got to my name in which I raised my hand quickly.\newline\newline
2) Which of the following is not a student's name in Matt's class?\\
A) Linda\\
B) Jason\\
\textbf{C) Frank}\\
D) Sally
% {\em Q:} Father asked if anyone wanted chicken on the pizza. Sue did not want chicken. Andy wanted chicken. Dan did not want chicken, but their father wanted to get chicken on the pizza.\newline\newline
% 3) Who did not want chicken on the pizza?\\
% A) Andy, Sue and Dan\\
% \textbf{B) Sue and Dan}\\
% C) Just Sue\\
% D) Just Dan
\end{flushleft}
\end{framed}
\caption{\label{fig:exampleMC500hard} A question from the MC500 test set.}
\end{figure}

\section{Future work and conclusion}
% Ellery ? Tom ? Emil?
%1/2 a page

%Semantic overlap is typically a symmetric relation while textual entailment is clearly not, this is a serious limitation of our baseline and the systems built on top. However, the great results show how really simple methods can achieve great results on the MCTest.
%
%hey i wrote a bit in the hypernym section about how i tried synonymy and it didn't work - the stories have such a small vocabulary its just noise
%Usage of synonyms as well as hypernym. Use Word2vec or Wordnet to achieve this.
%
% also i tried BOW and BOW w/ sentence selection on my rule based thing and it wasn't as good as sliding window
%A clear next step is to combine the rule-based system with the existing improved bag-of-words.
%
% i was thinking that the conclusion is: deeper methods are needed to solve mctest, since our system is almost at the limit of BOW matching. Tom was supposed to run an RTE system that would show this - idk if he did it though


The baseline scoring function that we construct and the subsequent improvements that we make to it are clearly pushing the limits of a simple lexical-based algorithm, with these techniques being better suited to a textual similarity task than recognising textual entailment. Our rule-based system does however demonstrate some potential, and further work to explore its use could consist of using more sophisticated transformations to the story text in combination with a stronger baseline algorithm than our Sliding Window with Distance algorithm. Our analysis using the rule-based system does show that MC500 poses much more of a challenge to our shallow approach than MC160 does, and it is likely that in order to make significant improvements with MC500 a deeper understanding of natural language will be needed.

\section*{Acknowledgments}
Thanks to Andreas Vlachos for his guidance throughout all the processes of this paper.

% include your own bib file like this:
\bibliographystyle{acl}
\bibliography{ref}
\end{document}
