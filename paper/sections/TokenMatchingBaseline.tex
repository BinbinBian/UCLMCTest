We propose a baseline scoring function, a simpler version compared to system proposed by \newcite{mctest} that only uses token matching (algorithm in Figure~\ref{fig:algoBOW}). Our system matches a bag-of-words constructed from the question and a candidate answer with each sentence in the story. The question-answer pair with the most words in common with a sentence in the text is considered the best candidate answer. Normalization are applied such as removing stop words stemming to remove affixes from words.

\begin{figure}[!th]
\subtitle Definitions

Passage $P$, word tokens in $j$-th sentence $P_j$, word tokens in question $Q$, word tokens in candidate answers $A_{1\ldots4}$ and set of stop words $X$.

\subtitle Algorithm 1 Sentence level bag-of-words

\begalg
def $TokenMatchingScore(P, Q, A)$:
  $score$ = $0$
  for $j$ = $0\ldots|P|-1$:
    $tmp$ = $(A_i\cup Q) \cap P_j \setminus X$
    if $tmp$ > $score$:
      $score$ = $tmp$
  return $score$
\endalg
\caption{\label{fig:algoBOW} Lexical-based baseline algorithm }
\end{figure}

This semantic overlap approach treats the problem as a textual similarity task and would perform best when the pair is a subset of a sentence. However, this strategy ignores predicate-argument structure and can easily fail in the presence of quantifiers, negations or synonyms. Work on story comprehension using bag-of-words has a long history, \newcite{deepread} proposed D{\small EEP}R{\small EAD} and showed how such systems with some heuristics can achieve high accuracy especially on questions with ``who / what / when", which is most part of our questions.

Results for MC160 and MC500 are shown in Table \ref{table:resultBOW}. The token-matching baseline has been authored without seeing both the test sets. The results are split for questions that need one sentence (``Single") or more sentence (``Multi") to answer the question. The 15\% drop between ``Single" and ``Multi" is due to choosing the most matching sentence in the text, hence multiple-sentence questions are clearly disadvantaged.

\begin{table}[!th]
\centering
\begin{tabular}{c|c|c|}
\cline{2-3}
\textbf{}                             & \textbf{MC160} & \textbf{MC500} \\ \hline
\multicolumn{1}{|c|}{\textit{Single}} & 67.88\%        & 62.763\%        \\ \hline
\multicolumn{1}{|c|}{\textit{Multi}}  & 50.31\%        & 46.72\%        \\ \hline
\multicolumn{1}{|c|}{\textit{All}}    & 58.43\%        & 53.97\%        \\ \hline
\end{tabular}
\caption{\label{table:resultBOW}Dev+train results for the bag-of-words baseline augmented with hypernym}
\end{table}
