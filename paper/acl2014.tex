%
% File acl2014.tex
%
% Contact: koller@ling.uni-potsdam.de, yusuke@nii.ac.jp
%%
%% Based on the style files for ACL-2013, which were, in turn,
%% Based on the style files for ACL-2012, which were, in turn,
%% based on the style files for ACL-2011, which were, in turn, 
%% based on the style files for ACL-2010, which were, in turn, 
%% based on the style files for ACL-IJCNLP-2009, which were, in turn,
%% based on the style files for EACL-2009 and IJCNLP-2008...

%% Based on the style files for EACL 2006 by 
%%e.agirre@ehu.es or Sergi.Balari@uab.es
%% and that of ACL 08 by Joakim Nivre and Noah Smith

\documentclass[11pt]{article}
\usepackage{acl2014}
\usepackage{times}
\usepackage{url}
\usepackage{latexsym}

%\setlength\titlebox{5cm}

% You can expand the titlebox if you need extra space
% to show all the authors. Please do not make the titlebox
% smaller than 5cm (the original size); we will check this
% in the camera-ready version and ask you to change it back.


\title{Solving MCTest with semantic textual similarity and matching rules??}

\author
       {A. Vlachos, T. Brown, N. Greco, G. Mocanu and E. J. Smith
       \\
       Computer Science Department\\
	University College London\\
       \tt{\{a.vlachos,t.brown,n.greco,g.mocanu,e.smirth\}@cs.ucl.ac.uk}\\ 
       }

\date{}

\begin{document}
\maketitle
\begin{abstract}
MCTest is a recently developed test for evaluating Machine Comprehension. Our approach to the task is through textual similarity and shallow methods. We build a simple bag-of-words baseline and we enhance it through additional pre-processing (i.e. coreference resolution, hypernym). We build a set of features and train a logistic classifier that we use to score multiple-choice answers. Finally we show how the introduction of simple matching rules system outperform current results on the MCTest.
\end{abstract}

\section{Introduction}
% Emil ? % Tom?
1 page \cite{mctest} and \newcite{mctest}

\section{Previous work}
% Nicola X % Tom ? % Emil ?
1 page

\section{Task description}
% Tom ? % Emil ?
1/2 a page--

\section{Baseline}
% Nicola X % Ellery ?
Simpler compared to sliding window.

However only textual similarity

half a page
\section{Preprocessing}
2 pages
\subsection{Co-reference}
% Emil ? % Ellery ?
\subsection{Hypernym}
% Ellery ?
\subsection{Sentence selection}
% Nicola X
\subsection{Question answer combiner}
% Tom?

\section{Strategy}
1.5 pages
\subsection{Bag of words}
% Nicola X % Ellery ?
several features
BOWNN
BOW Complement
Bow ALL

\subsection{Scoring function}
% Nicola X
same problem as Deep Selection

\subsection{Rule based system}
% Ellery X

\section{Experiments and results}
% Nicola ? % Emil ? % Tom?
1 page

\section{Evaluation of strategies}
% Ellery ?
3 pages

\section{Future work and conclusion}
% Ellery ? Tom ? Emil?
1/2 a page

\section*{Acknowledgments}
Thanks to

% include your own bib file like this:
\bibliographystyle{acl}
\bibliography{ref}
\end{document}
