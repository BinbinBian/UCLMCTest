Co-reference refers to extracting relations between expressions that refer to the same entity. In most cases when we have a co-reference relation one of the expressions is the full form of the referred object, called the antecedent and the other is the abbreviated form, which can take many forms. For our system having this information is useful as we can replace the abbreviated form with the full form and obtain more useful sentences to match against. 

The co-reference information was extracted using the Stanford CoreNLP \cite{manning2014stanford}. The passage text was parsed independently of the question and answer strings, so all co-reference chains were local to the story itself. We used an out-of-the-box configuration of the co-reference rules, as this was deemed to perform adequately on the simplistic format of the given stories. Using co-reference on the question answer pairs, or on the story text plus the question answer pairs proved to be detrimental to performance.